\documentclass[12pt, a4paper, twoside]{report}
\usepackage[utf8]{inputenc}
\usepackage[catalan]{babel}
\usepackage[a4paper, portrait, margin=1in]{geometry}
\usepackage[T1]{fontenc}

\setlength{\parindent}{0pt}
\setlength{\parskip}{\baselineskip}


\title{Física}
\author{Martí Bravo Navarro}
\date{Quadrimestre de primavera 2021}

\pagestyle{headings}
\begin{document}
\begin{titlepage}
	\centering
	\vspace*{\baselineskip} 
    {\scshape Universitat Politècnica de Catalunya}
    \vspace{0.25\baselineskip}

    {\scshape\Large Facultat d'Informàtica de Barcelona}
	\vspace{10\baselineskip}

    \textbf{\Large Apunts}
    \vspace{0.75\baselineskip}

	{\LARGE FÍSICA\\}

	\vspace{0.75\baselineskip}
	\vspace{10\baselineskip}
    Quadrimestre de Primavera 2021
	\vspace*{3\baselineskip} 

    \textit{per}
	\vspace{0.5\baselineskip}
	{\scshape\Large Martí Bravo Navarro}
	\vspace{0.5\baselineskip}
	\vfill
	

\end{titlepage}

\section*{Avaluació}
\begin{itemize}
    \item \textbf{Teoria} --- 90\%
    \begin{itemize}
        \item Els parcials es poden reeditar al final, que consta de dues parts, una corresponent a cada parcial. 
            \item \textbf{45\%} Parcial 1 / Final 1
        \begin{itemize}
            \item Temes 1 i 2. Corrent Continu i Corrent Altern
        \end{itemize}
        \item \textbf{45\%} Parcial 2 / Final 2
        \begin{itemize}
            \item Temes 3 i 4. Electrònica, portes lògiques i ones.     
        \end{itemize}
    \end{itemize}
    \item \textbf{Laboratoris}   --- 10\% --- 5 laboratoris + Presentació
    \begin{itemize}
        \item \textbf{7,5\%} Mitja de totes les pràctiques
        \begin{itemize}
            \item Treball previ a cada laboratori
            \item Cal anar al laboratori
            \item Informe final entregat al Racó.
        \end{itemize}
        \item \textbf{2,5\%} Presentació oral sobre una pràctica. 
        
    \end{itemize}
  \end{itemize}

\section*{Altres}
\subsection*{Tallers}
Abans de cada final es faran tallers voluntaris i presencials per preguntar dubtes als professors. 

\chapter{Corrent continu}
\section{Càrrega elèctrica}

La càrrega elèctrica \(q\) o \(Q\)  és una propietat dels objectes que presenten interaccions elèctriques.
Si un objecte té interaccions elèctriques té càrrega elèctrica.
Una càrrega pot ser \textbf{positiva o negativa}
Les interaccions elèctriques poden ser:
\begin{itemize}
    \item Atractives. Hi intervenen càrregues de signe contrari.
    \item Repulsives. Les càrregues són de mateix signe.
\end{itemize} 

La càrrega \textbf{està quantitzada} (\(q = \pm Ne\)).

La càrrega \textbf{es conserva}. Dues molècules amb \(q_1, q_2\) interaccionen i acaben amb càrregues \(q_1', q_2'\), on la suma de càrrega als dos casos és la mateixa.

La càrrega d'un electró és \(q_e = -e \), on \(e = 1,6 * 10^-19 \))

\section{Camp elèctric i potencial}
Camp elèctric: \overrightarrow{E}
Potencial elèctric: \(V\) 

Les interaccions electrostàtiques són forces a distància.
$$ q \rightarrow \overrightarrow{F} \equiv q \rightarrow \overrightarrow{F} $$
$$\overrightarrow{F} =  q \overrightarrow{E} \Longrightarrow \overrightarrow{E} = \frac{\overrightarrow{F}}{q} $$
$$ q = 1C \rightarrow \overrightarrow{E} = \overrightarrow{F}$$ en Newtons/Coloumb

Les interaccions electrostàtiques són forces conservatives. 
Energia potencial electrostàtica \(u\), unitats en Joule, \(J\). 
\(u \equiv\) treball necessari per traslladar \(u\) a aquest punt. 

$$ U \equiv qV \longrightarrow V = \frac{U}{q} $$

V \(\longrightarrow\) unitats en Volts, V. \(1V = \frac{1J}{1C}\)

\section{Corrent elèctric. Intensitat.}
El flux o moviment de càrregues elèctriques en un medi. \textbf{Exemple:} Corrent elèctric en un fil conductor (de coure Cu) amb una secció \(S\)
$$I = \frac{dQ}{dt}$$
En Amperes, A. \(1A = \frac{1C}{s}\)

La Intensitat \(I\) també és la càrrega que travessa una secció S per unitat de temps. 

Càrregues en moviment
\begin{itemize}
    \item \(q\): càrrega d'un portador
    \item \(\mathcal{V}_d\): velocitat de deriva
    \item \(n\): densitat de portadors de càrrega
    \item \mu mobilitat de les càrregues en el material. 
\end{itemize}
$$dQ = (n \ S  \ \mathcal{V}_d \ dt) \longrightarrow I = | q \vert \ n \ S  \ \mathcal{V}_d  $$
$$ \overrightarrow{\mathcal{V}_d} = \pm \mu \overrightarrow{E} $$ 
\subsubsection{Densitat de corrent \textbf{J}}
\begin{itemize}
    \item Si \(q > 0\), \overrightarrow{d} té mateix sentit que la velocitat de deriva. \overrightarrow{F} mateix sentit que \overrightarrow{E}. 
    \item Si \(q < 0\), \overrightarrow{d} té sentit contrari que la velocitat de deriva. \overrightarrow{F} sentit contrari que \overrightarrow{E}
\end{itemize}
$$J = \frac{I}{S} = | q \vert \ n \ \mathcal{V}_d \ \ \overrightarrow{J} = q \ n \ \overrightarrow{\mathcal{V}_d}$$
\chapter{Corrent altern}
\chapter{Electrònica i portes lògiques}
\chapter{Ones}
\end{document}